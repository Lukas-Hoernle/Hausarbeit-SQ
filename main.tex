\documentclass{article}
\usepackage{biblatex}
\bibliography{quellen}
\begin{document}
\title{Analyse, Bewertung und Optimierung der Softwarequalität im Kontext eines Online-Shops für Elektronikprodukte}
\author{Lukas Hörnle}
\maketitle

\tableofcontents
\newpage

\section{Einleitung}
Die Qualität von Software ist in der heutigen digitalisierten Welt, besonders im E-Commerce-Bereich, von großer Bedeutung. Angesichts des wachsenden Wettbewerbs und der zunehmenden Bedeutung von Online-Shops als Vertriebskanal für elektronische Produkte ist es entscheidend, eine hohe Softwarequalität sicherzustellen. Mangelhafte Softwarequalität kann nicht nur finanzielle Verluste verursachen, sondern auch das Vertrauen der Kunden in den Online-Shop nachhaltig schädigen \footfullcite{pressman2014software}.

Das Ziel dieser Hausarbeit besteht darin, anhand eines praktischen Fallbeispiels eines realen Online-Shops für Elektronikprodukte die Softwarequalität zu analysieren, zu bewerten, kritisch zu hinterfragen und Optimierungsmaßnahmen vorzuschlagen. Dabei wird die theoretische Konzeption der Softwarequalität auf eine konkrete Situation angewendet. Das Fallbeispiel konzentriert sich auf die Qualitätssicherung und Performance-Optimierung des Online-Shops, insbesondere auf das Problem der langen Ladezeiten der Produktseiten.

Im weiteren Verlauf dieser Arbeit werden zunächst die theoretischen Grundlagen der Softwarequalität erläutert. Dies beinhaltet eine präzise Definition von "Softwarequalität", die Vorstellung relevanter Kriterien zur Bewertung der Qualität sowie eine Übersicht über verschiedene Methoden und Modelle zur Sicherung der Softwarequalität \footfullcite{sommerville2016software}$^{,\hspace{1pt}}$\footfullcite{pfleeger2010software}. Anschließend wird das praktische Fallbeispiel des Online-Shops für Elektronikprodukte eingeführt, wobei der Fokus auf dem Aspekt der Softwarequalität liegt.

Die Analyse der Softwarequalität im Fallbeispiel konzentriert sich auf die Identifizierung eines kritischen Punktes, nämlich der Performance-Probleme beim Laden der Produktseiten. Dabei werden das Problem und seine weitreichenden Auswirkungen auf die Gesamtsituation des Online-Shops detailliert beschrieben.

Eine Bewertung der aktuellen Softwarequalität im Fallbeispiel erfolgt im Vergleich zu etablierten Methoden zur Performance-Optimierung. Dabei werden die Auswirkungen der aktuellen Softwarequalität auf die Gesamtsituation des Online-Shops analysiert und bewertet.

Des Weiteren wird eine kritische Betrachtung der aktuellen Softwarequalität im Fallbeispiel durchgeführt, um die Ursachen der Performance-Probleme zu analysieren und potenzielle Schwachstellen zu identifizieren \footfullcite{basili1994goal}.

Abschließend werden Optimierungsmaßnahmen zur Verbesserung der Performance vorgeschlagen, wie z.B. die Optimierung des Datenbankzugriffs und die Implementierung von Caching-Strategien für häufig angeforderte Daten. Die Auswirkungen dieser Maßnahmen auf die Gesamtsituation des Online-Shops werden ebenfalls betrachtet.

In den Schlussfolgerungen werden die Ergebnisse der Analyse, Bewertung, kritischen Betrachtung und Optimierungsmaßnahmen zusammengefasst und mögliche zukünftige Entwicklungen diskutiert.
\section{Theoretische Grundlagen}
\subsection{Softwarequalität: Definition und Bedeutung}
\subsection{Kriterien der Softwarequalität}
\subsection{Methoden und Modelle zur Softwarequalitätssicherung}

\section{Praktisches Fallbeispiel: Qualitätssicherung eines Online-Shops für Elektronikprodukte}
\subsection{Beschreibung des Fallbeispiels}
\subsection{Betrachtung des SQMs im Fallbeispiel}

\section{Analyse der Softwarequalität im Fallbeispiel}
\subsection{Identifizierung des kritischen Punktes: Performance-Probleme beim Laden der Produktseiten}
\subsubsection{Beschreibung des Problems}
\subsubsection{Auswirkungen auf die Gesamtsituation}

\section{Bewertung der Softwarequalität im Fallbeispiel}
\subsection{Vergleich mit etablierten Methoden zur Performance-Optimierung}
\subsection{Analyse der Auswirkungen auf die Gesamtsituation}

\section{Kritik der aktuellen Softwarequalität im Fallbeispiel}
\subsection{Ursachenanalyse der Performance-Probleme}
\subsection{Schwachstellen der aktuellen Softwarequalität}

\section{Erarbeitung von Optimierungsmaßnahmen}
\subsection{Maßnahmen zur Verbesserung der Performance}
\subsubsection{Optimierung des Datenbankzugriffs}
\subsubsection{Implementierung von Caching-Strategien für häufig angeforderte Daten}
\subsection{Auswirkungen der Optimierungsmaßnahmen auf die Gesamtsituation}

\section{Schlussfolgerungen und Ausblick}
\subsection{Zusammenfassung der Analyse, Bewertung, Kritik und Optimierung}
\subsection{Bedeutung der praktischen Anwendung von Softwarequalität im Kontext von E-Commerce}
\subsection{Ausblick auf zukünftige Entwicklungen im Bereich der Softwarequalität im E-Commerce}

\section{Literaturverzeichnis}
\printbibliography[title={quellen.tex}]
\end{document}
