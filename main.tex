\documentclass{article}
\usepackage{biblatex}
\bibliography{quellen}
\begin{document}
\title{Analyse, Bewertung und Optimierung der Softwarequalität im Kontext eines Online-Shops für Elektronikprodukte}
\author{Lukas Hörnle}
\maketitle

\tableofcontents
\newpage

\section{Einleitung}
Die Qualität von Software ist ein entscheidender Faktor in der heutigen digitalisierten Welt, insbesondere im Bereich des E-Commerce. Angesichts des zunehmenden Wettbewerbs und der wachsenden Bedeutung von Online-Shops als Vertriebskanal für elektronische Produkte ist es von zentraler Bedeutung, eine hohe Softwarequalität sicherzustellen. Eine mangelhafte Qualität der eingesetzten Software kann nicht nur zu finanziellen Verlusten führen, sondern auch das Vertrauen der Kunden in den Online-Shop nachhaltig beeinträchtigen.

Das Ziel dieser Hausarbeit besteht darin, anhand eines praktischen Fallbeispiels eines realen Online-Shops für Elektronikprodukte die Softwarequalität zu analysieren, zu bewerten, kritisch zu hinterfragen und Optimierungsmaßnahmen vorzuschlagen. Hierbei wird die theoretische Konzeption der Softwarequalität auf eine konkrete Situation angewendet. Das Fallbeispiel konzentriert sich dabei auf die Qualitätssicherung und die Performance-Optimierung des Online-Shops, insbesondere auf die Problematik der langen Ladezeiten der Produktseiten.

Im weiteren Verlauf dieser Arbeit werden zunächst die theoretischen Grundlagen der Softwarequalität erläutert. Dies beinhaltet eine präzise Definition des Begriffs "Softwarequalität", die Vorstellung relevanter Kriterien zur Beurteilung der Qualität sowie eine Übersicht über verschiedene Methoden und Modelle zur Gewährleistung der Softwarequalität. Im Anschluss wird das praktische Fallbeispiel des Online-Shops für Elektronikprodukte eingeführt, wobei der Fokus speziell auf den Aspekt der Softwarequalität gelegt wird.

Die Analyse der Softwarequalität im Fallbeispiel konzentriert sich auf die Identifizierung eines kritischen Punktes, nämlich der Performance-Probleme beim Laden der Produktseiten. In diesem Zusammenhang werden das Problem und seine weitreichenden Auswirkungen auf die Gesamtsituation des Online-Shops detailliert beschrieben.

Anschließend erfolgt eine Bewertung der aktuellen Softwarequalität im Fallbeispiel, wobei eine Gegenüberstellung mit etablierten Methoden zur Performance-Optimierung erfolgt. In diesem Zusammenhang werden die Auswirkungen der derzeitigen Softwarequalität auf die Gesamtsituation des Online-Shops analysiert und bewertet.

Des Weiteren wird eine kritische Betrachtung der aktuellen Softwarequalität im Fallbeispiel vorgenommen, um die Ursachen der Performance-Probleme zu analysieren und potenzielle Schwachstellen zu identifizieren.

Abschließend werden Optimierungsmaßnahmen zur Verbesserung der Performance vorgeschlagen. Diese Maßnahmen umfassen beispielsweise die Optimierung des Datenbankzugriffs sowie die Implementierung von Caching-Strategien für häufig angeforderte Daten. Die Auswirkungen dieser Optimierungsmaßnahmen auf die Gesamtsituation des Online-Shops werden ebenfalls betrachtet.

In den Schlussfolgerungen werden die Ergebnisse der Analyse, Bewertung, kritischen Betrachtung und

\section{Theoretische Grundlagen}

\subsection{Softwarequalität: Definition und Bedeutung}
Die Softwarequalität spielt eine entscheidende Rolle in der Entwicklung und dem Einsatz von Software. Sie bezieht sich auf die Fähigkeit einer Software, die Anforderungen und Erwartungen der Benutzer zu erfüllen. Die Softwarequalität umfasst verschiedene Aspekte wie Funktionalität, Zuverlässigkeit, Effizienz, Benutzerfreundlichkeit, Wartbarkeit und Portabilität. Eine hohe Softwarequalität ist von großer Bedeutung, um die Zufriedenheit der Benutzer zu gewährleisten, die Zuverlässigkeit und Sicherheit der Software zu verbessern und die Wartungskosten zu reduzieren \footfullcite{pressman2014software}$^{,\hspace{1pt}}$\footfullcite{sommerville2016software}.

\subsection{Kriterien der Softwarequalität}
Bei der Beurteilung der Softwarequalität werden verschiedene Kriterien herangezogen. Dazu gehören:

\begin{itemize}
\item Funktionalität: Die Software muss die definierten Funktionen erfüllen und die gewünschten Ergebnisse liefern.
\item Zuverlässigkeit: Die Software sollte in der Lage sein, fehlerfrei zu arbeiten und konsistente Ergebnisse zu liefern.
\item Effizienz: Die Software sollte die verfügbaren Ressourcen effizient nutzen, um optimale Leistung zu erzielen.
\item Benutzerfreundlichkeit: Die Software sollte leicht verständlich und einfach zu bedienen sein, um die Benutzerzufriedenheit zu fördern.
\item Wartbarkeit: Die Software sollte leicht anpassbar und erweiterbar sein, um Änderungen und zukünftige Anforderungen zu unterstützen.
\item Portabilität: Die Software sollte auf verschiedenen Plattformen und Umgebungen einsetzbar sein, um die Flexibilität und Skalierbarkeit zu gewährleisten \footfullcite{pfleeger2010software}.
\end{itemize}

Die genannten Kriterien dienen als Maßstab für die Bewertung der Softwarequalität und helfen bei der Identifizierung von Verbesserungspotenzialen.

\subsection{Methoden und Modelle zur Softwarequalitätssicherung}
Zur Gewährleistung der Softwarequalität stehen verschiedene Methoden und Modelle zur Verfügung. Ein weit verbreitetes Modell ist das V-Modell, das den gesamten Softwareentwicklungsprozess von der Anforderungsanalyse bis zur Wartung abdeckt und klare Phasen und Aktivitäten definiert. Ein weiteres bekanntes Modell ist das Wasserfallmodell, das einen sequentiellen Ansatz verfolgt und den Entwicklungsprozess in klar definierte Phasen unterteilt \footfullcite{pressman2014software}$^{,\hspace{1pt}}$\footfullcite{sommerville2016software}.

Darüber hinaus gibt es verschiedene Qualitätsmanagementmethoden wie das Goal Question Metric (GQM)-Modell, das eine strukturierte Herangehensweise an die Messung und Bewertung von Softwarequalität bietet. Das GQM-Modell basiert auf der Festlegung von Zielen, der Formulierung von Fragen und der Auswahl geeigneter Metriken zur Bewertung der Softwarequalität \footfullcite{basili1994goal}.

Die Softwarequalitätssicherung umfasst auch die Durchführung von Tests, um die Funktionalität und Zuverlässigkeit der Software zu überprüfen. Dies umfasst Unit-Tests, Integrationstests, Systemtests und Akzeptanztests \footfullcite{ieee730}.

Die Auswahl der geeigneten Methoden und Modelle zur Softwarequalitätssicherung hängt von den spezifischen Anforderungen und dem Kontext des Softwareprojekts ab.

\section{Praktisches Fallbeispiel: Qualitätssicherung eines Online-Shops für Elektronikprodukte}
\subsection{Beschreibung des Fallbeispiels}
\subsection{Betrachtung des SQMs im Fallbeispiel}

\section{Analyse der Softwarequalität im Fallbeispiel}
\subsection{Identifizierung des kritischen Punktes: Performance-Probleme beim Laden der Produktseiten}
\subsubsection{Beschreibung des Problems}
\subsubsection{Auswirkungen auf die Gesamtsituation}

\section{Bewertung der Softwarequalität im Fallbeispiel}
\subsection{Vergleich mit etablierten Methoden zur Performance-Optimierung}
\subsection{Analyse der Auswirkungen auf die Gesamtsituation}

\section{Kritik der aktuellen Softwarequalität im Fallbeispiel}
\subsection{Ursachenanalyse der Performance-Probleme}
\subsection{Schwachstellen der aktuellen Softwarequalität}

\section{Erarbeitung von Optimierungsmaßnahmen}
\subsection{Maßnahmen zur Verbesserung der Performance}
\subsubsection{Optimierung des Datenbankzugriffs}
\subsubsection{Implementierung von Caching-Strategien für häufig angeforderte Daten}
\subsection{Auswirkungen der Optimierungsmaßnahmen auf die Gesamtsituation}

\section{Schlussfolgerungen und Ausblick}
\subsection{Zusammenfassung der Analyse, Bewertung, Kritik und Optimierung}
\subsection{Bedeutung der praktischen Anwendung von Softwarequalität im Kontext von E-Commerce}
\subsection{Ausblick auf zukünftige Entwicklungen im Bereich der Softwarequalität im E-Commerce}

\section{Literaturverzeichnis}
\printbibliography[title={quellen.tex}]
\end{document}
