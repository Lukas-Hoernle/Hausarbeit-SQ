\documentclass{article}
\begin{document}
\title{Analyse, Bewertung und Optimierung der Softwarequalität im Kontext eines Online-Shops für Elektronikprodukte}
\author{Lukas Hörnle}
\maketitle

\tableofcontents

\section{Einleitung}
\section{Theoretische Grundlagen}
\subsection{Softwarequalität: Definition und Bedeutung}
\subsection{Kriterien der Softwarequalität}
\subsection{Methoden und Modelle zur Softwarequalitätssicherung}

\section{Praktisches Fallbeispiel: Qualitätssicherung eines Online-Shops für Elektronikprodukte}
\subsection{Beschreibung des Fallbeispiels}
\subsection{Betrachtung des SQMs im Fallbeispiel}

\section{Analyse der Softwarequalität im Fallbeispiel}
\subsection{Identifizierung des kritischen Punktes: Performance-Probleme beim Laden der Produktseiten}
\subsubsection{Beschreibung des Problems}
\subsubsection{Auswirkungen auf die Gesamtsituation}

\section{Bewertung der Softwarequalität im Fallbeispiel}
\subsection{Vergleich mit etablierten Methoden zur Performance-Optimierung}
\subsection{Analyse der Auswirkungen auf die Gesamtsituation}

\section{Kritik der aktuellen Softwarequalität im Fallbeispiel}
\subsection{Ursachenanalyse der Performance-Probleme}
\subsection{Schwachstellen der aktuellen Softwarequalität}

\section{Erarbeitung von Optimierungsmaßnahmen}
\subsection{Maßnahmen zur Verbesserung der Performance}
\subsubsection{Optimierung des Datenbankzugriffs}
\subsubsection{Implementierung von Caching-Strategien für häufig angeforderte Daten}
\subsection{Auswirkungen der Optimierungsmaßnahmen auf die Gesamtsituation}

\section{Schlussfolgerungen und Ausblick}
\subsection{Zusammenfassung der Analyse, Bewertung, Kritik und Optimierung}
\subsection{Bedeutung der praktischen Anwendung von Softwarequalität im Kontext von E-Commerce}
\subsection{Ausblick auf zukünftige Entwicklungen im Bereich der Softwarequalität im E-Commerce}

\section{Literaturverzeichnis}

\end{document}
