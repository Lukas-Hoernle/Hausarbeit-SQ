\section{Praktisches Fallbeispiel: Qualitätssicherung eines Online-Shops für Elektronikprodukte}

\subsection{Beschreibung des Fallbeispiels}
Das Fallbeispiel betrifft einen Online-Shop für Elektronikprodukte, betrieben von der Beispiel AG. Der Shop bietet eine breite Auswahl an elektronischen Produkten wie Mobiltelefonen, Laptops, Kameras und Zubehör an. Angesichts der wachsenden Bedeutung des E-Commerce im Elektronikbereich ist es für die Beispiel AG von großer Bedeutung, eine hohe Softwarequalität sicherzustellen, um Kunden zufriedenzustellen und wettbewerbsfähig zu bleiben.

Der Online-Shop der Beispiel AG zeichnet sich durch eine benutzerfreundliche Oberfläche, eine große Produktauswahl und attraktive Angebote aus. Kunden können Produkte durchsuchen, auswählen, in den Warenkorb legen und den Bestellvorgang abschließen. Der Shop ist in verschiedene Kategorien und Unterkategorien unterteilt, um die Navigation und Produktsuche zu erleichtern. Zusätzlich werden Informationen zu den Produkten, Kundenbewertungen, Produktvergleiche und zusätzliche Services wie Garantie- und Reparaturoptionen angeboten.

\subsection{Betrachtung des SQMs im Fallbeispiel}
Im Rahmen der Qualitätssicherung des Online-Shops der Beispiel AG wird der Softwarequalitätsmaßstab (SQM) verwendet, um die Qualität der eingesetzten Software zu bewerten. Der SQM umfasst verschiedene Kriterien, die bei der Beurteilung der Softwarequalität berücksichtigt werden.

Zunächst wird die Funktionalität des Online-Shops analysiert. Dabei werden die Erfüllung der definierten Funktionen, die korrekte Darstellung von Produktinformationen, die Verfügbarkeit von Such- und Filterfunktionen sowie die reibungslose Abwicklung des Bestellvorgangs überprüft. Auch die Zuverlässigkeit des Shops wird bewertet, indem Fehlerfreiheit, Konsistenz und Verfügbarkeit der Dienste analysiert werden.

Die Effizienz des Online-Shops wird anhand der Ladezeiten der Produktseiten bewertet. Lange Ladezeiten können zu einer negativen Benutzererfahrung führen und potenzielle Kunden abschrecken. Daher werden Performance-Probleme identifiziert und analysiert, um Optimierungsmöglichkeiten vorzuschlagen.

Auch die Benutzerfreundlichkeit des Online-Shops spielt eine wichtige Rolle. Aspekte wie intuitive Navigation, Verständlichkeit der Benutzeroberfläche, Präsentation von Produktinformationen und Unterstützung von Kundenbewertungen werden bewertet.

Die Wartbarkeit des Online-Shops wird analysiert, um sicherzustellen, dass er leicht anpassbar und erweiterbar ist, um Änderungen und zukünftigen Anforderungen gerecht zu werden. Dazu gehören die Überprüfung der Codequalität, Verwendung bewährter Entwicklungspraktiken und die Dokumentation des Quellcodes.

Abschließend wird die Portabilität des Online-Shops betrachtet, um sicherzustellen, dass er auf verschiedenen Plattformen und in verschiedenen Umgebungen eingesetzt werden kann. Dies beinhaltet die Überprüfung der Kompatibilität mit verschiedenen Webbrowsern, Betriebssystemen und Bildschirmauflösungen.

Durch die Anwendung des SQMs werden Schwachstellen und Verbesserungspotenziale in Bezug auf die Softwarequalität des Online-Shops der Beispiel AG identifiziert. Auf dieser Grundlage können gezielte Optimierungsmaßnahmen entwickelt werden, um die Qualität des Online-Shops zu verbessern und die Kundenzufriedenheit zu steigern.\footfullcite{drees1999qualitatssicherung}