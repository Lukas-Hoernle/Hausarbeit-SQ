\section{Praktisches Fallbeispiel: Qualitätssicherung eines Online-Shops für Elektronikprodukte}

\subsection{Beschreibung des Fallbeispiels}
Das praktische Fallbeispiel konzentriert sich auf die Qualitätssicherung eines Online-Shops für Elektronikprodukte, der von der Beispiel AG betrieben wird. Der Online-Shop bietet eine breite Palette von elektronischen Produkten wie Mobiltelefone, Laptops, Kameras und Zubehör an. Angesichts der wachsenden Bedeutung des E-Commerce als Vertriebskanal für elektronische Produkte ist es für die Beispiel AG von entscheidender Bedeutung, eine hohe Softwarequalität sicherzustellen, um Kunden zufriedenzustellen und ihre Konkurrenzfähigkeit zu gewährleisten.

Der Online-Shop der Beispiel AG zeichnet sich durch eine benutzerfreundliche Oberfläche, eine große Produktauswahl und attraktive Angebote aus. Kunden können Produkte durchsuchen, auswählen, in den Warenkorb legen und den Bestellvorgang abschließen. Der Online-Shop ist in verschiedene Kategorien und Unterkategorien unterteilt, um die Navigation und Suche nach Produkten zu erleichtern. Darüber hinaus bietet der Online-Shop Informationen zu den Produkten, Kundenbewertungen, Produktvergleiche und zusätzliche Services wie Garantie- und Reparaturoptionen.

\subsection{Betrachtung des SQMs im Fallbeispiel}
Im Rahmen der Qualitätssicherung des Online-Shops für Elektronikprodukte der Beispiel AG wird der Softwarequalitätsmaßstab (SQM) angewendet, um die Qualität der eingesetzten Software zu bewerten. Der SQM umfasst verschiedene Kriterien, die bei der Beurteilung der Softwarequalität berücksichtigt werden.

Zunächst wird die Funktionalität des Online-Shops analysiert. Hierbei werden die Erfüllung der definierten Funktionen, die korrekte Darstellung von Produktinformationen, die Verfügbarkeit von Such- und Filterfunktionen sowie die reibungslose Abwicklung des Bestellvorgangs überprüft. Darüber hinaus wird die Zuverlässigkeit des Online-Shops bewertet, indem die Fehlerfreiheit, Konsistenz und Verfügbarkeit der Dienste analysiert werden.

Die Effizienz des Online-Shops wird anhand der Ladezeiten der Produktseiten bewertet. Lange Ladezeiten können zu einer negativen Benutzererfahrung führen und potenzielle Kunden abschrecken. Daher werden die Performance-Probleme beim Laden der Produktseiten identifiziert und analysiert, um mögliche Optimierungsmaßnahmen vorzuschlagen.

Die Benutzerfreundlichkeit des Online-Shops spielt ebenfalls eine wichtige Rolle. Hierbei werden Aspekte wie die intuitive Navigation, die Verständlichkeit der Benutzeroberfläche, die Präsentation von Produktinformationen und die Unterstützung von Kundenbewertungen bewertet.

Die Wartbarkeit des Online-Shops wird analysiert, um sicherzustellen, dass der Shop leicht anpassbar und erweiterbar ist, um Änderungen und zukünftige Anforderungen zu unterstützen. Dies umfasst die Überprüfung der Codequalität, die Verwendung von bewährten Entwicklungspraktiken und die Dokumentation des Quellcodes.

Abschließend wird die Portabilität des Online-Shops betrachtet, um sicherzustellen, dass er auf verschiedenen Plattformen und Umgebungen einsetzbar ist. Dies beinhaltet die Überprüfung der Kompatibilität mit verschiedenen Webbrowsern, Betriebssystemen und Bildschirmauflösungen.

Durch die Anwendung des SQMs werden Schwachstellen und Verbesserungspotenziale im Bereich der Softwarequalität des Online-Shops für Elektronikprodukte der Beispiel AG identifiziert. Auf dieser Grundlage können gezielte Optimierungsmaßnahmen entwickelt werden, um die Qualität des Online-Shops zu verbessern und die Zufriedenheit der Kunden zu steigern.