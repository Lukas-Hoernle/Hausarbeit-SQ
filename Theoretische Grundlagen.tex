\section{Theoretische Grundlagen}

\subsection{Softwarequalität: Definition und Bedeutung}
Die Softwarequalität spielt eine entscheidende Rolle in der Entwicklung und dem Einsatz von Software. Sie bezieht sich auf die Fähigkeit einer Software, die Anforderungen und Erwartungen der Benutzer zu erfüllen. Die Softwarequalität umfasst verschiedene Aspekte wie Funktionalität, Zuverlässigkeit, Effizienz, Benutzerfreundlichkeit, Wartbarkeit und Portabilität. Eine hohe Softwarequalität ist von großer Bedeutung, um die Zufriedenheit der Benutzer zu gewährleisten, die Zuverlässigkeit und Sicherheit der Software zu verbessern und die Wartungskosten zu reduzieren \footfullcite{pressman2014software}$^{,\hspace{1pt}}$\footfullcite{sommerville2016software}.

\subsection{Kriterien der Softwarequalität}
Bei der Beurteilung der Softwarequalität werden verschiedene Kriterien herangezogen. Dazu gehören:

\begin{itemize}
\item Funktionalität: Die Software muss die definierten Funktionen erfüllen und die gewünschten Ergebnisse liefern.
\item Zuverlässigkeit: Die Software sollte in der Lage sein, fehlerfrei zu arbeiten und konsistente Ergebnisse zu liefern.
\item Effizienz: Die Software sollte die verfügbaren Ressourcen effizient nutzen, um optimale Leistung zu erzielen.
\item Benutzerfreundlichkeit: Die Software sollte leicht verständlich und einfach zu bedienen sein, um die Benutzerzufriedenheit zu fördern.
\item Wartbarkeit: Die Software sollte leicht anpassbar und erweiterbar sein, um Änderungen und zukünftige Anforderungen zu unterstützen.
\item Portabilität: Die Software sollte auf verschiedenen Plattformen und Umgebungen einsetzbar sein, um die Flexibilität und Skalierbarkeit zu gewährleisten \footfullcite{pfleeger2010software}.
\end{itemize}

Die genannten Kriterien dienen als Maßstab für die Bewertung der Softwarequalität und helfen bei der Identifizierung von Verbesserungspotenzialen.

\subsection{Methoden und Modelle zur Softwarequalitätssicherung}
Zur Gewährleistung der Softwarequalität stehen verschiedene Methoden und Modelle zur Verfügung. Ein weit verbreitetes Modell ist das V-Modell, das den gesamten Softwareentwicklungsprozess von der Anforderungsanalyse bis zur Wartung abdeckt und klare Phasen und Aktivitäten definiert. Ein weiteres bekanntes Modell ist das Wasserfallmodell, das einen sequentiellen Ansatz verfolgt und den Entwicklungsprozess in klar definierte Phasen unterteilt \footfullcite{pressman2014software}$^{,\hspace{1pt}}$\footfullcite{sommerville2016software}.

Darüber hinaus gibt es verschiedene Qualitätsmanagementmethoden wie das Goal Question Metric (GQM)-Modell, das eine strukturierte Herangehensweise an die Messung und Bewertung von Softwarequalität bietet. Das GQM-Modell basiert auf der Festlegung von Zielen, der Formulierung von Fragen und der Auswahl geeigneter Metriken zur Bewertung der Softwarequalität \footfullcite{basili1994goal}.

Die Softwarequalitätssicherung umfasst auch die Durchführung von Tests, um die Funktionalität und Zuverlässigkeit der Software zu überprüfen. Dies umfasst Unit-Tests, Integrationstests, Systemtests und Akzeptanztests \footfullcite{ieee730}.

Die Auswahl der geeigneten Methoden und Modelle zur Softwarequalitätssicherung hängt von den spezifischen Anforderungen und dem Kontext des Softwareprojekts ab.